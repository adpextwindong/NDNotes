\documentclass{article}
\usepackage[utf8]{inputenc}
\usepackage{bussproofs}
\usepackage{comment}

\title{}
\author{}
\date{}

\begin{document}

\maketitle
%https://homepages.inf.ed.ac.uk/wadler/papers/propositions-as-types/propositions-as-types.pdf
%http://math.ucsd.edu/~sbuss/ResearchWeb/bussproofs/BussGuide2_Smith2012.pdf
% https://tex.stackexchange.com/a/283586
% https://tex.stackexchange.com/a/278156
% https://tex.stackexchange.com/a/214014

\begin{center}
        
    \begin{tabular}{ ccc}
        
        \AxiomC{A}
        \AxiomC{B}
        \RightLabel{\&-I}
        \BinaryInfC{A\&B}
        \DisplayProof
        & 
            \AxiomC{A}
        \AxiomC{B}
        \RightLabel{$\&$-$E_1$}
        \BinaryInfC{A}
        \DisplayProof
        & 
        \AxiomC{A}
        \AxiomC{B}
        \RightLabel{$\&$-$E_2$}
        \BinaryInfC{B}
        \DisplayProof
    \end{tabular}
\end{center}
\begin{center}
    \begin{tabular}{ cc}
    
    \AxiomC{[$A^x$]}
    \noLine
    \UnaryInfC{$\vdots$}
    \noLine
    \UnaryInfC{B}
    \RightLabel{$\supset$-$I^x$}
    \UnaryInfC{$A \supset B$}
    \DisplayProof
    &
    \AxiomC{$A\supset B$}
    \AxiomC{A}
    \RightLabel{$\supset$-$E$}
    \BinaryInfC{B}
    \DisplayProof
\end{tabular}
\end{center}

Figure 1. Gerhard Gentzen (1935) - Natural Deduction - Proof Rules

\end{document}